\subsubsection{Ciclo de vida del producto}

Los productos Disney nacen en formato audiovisual como personajes de largometrajes, cortometrajes o series. Tras el estreno de la obra, en la que los consumidores obtienen una gran cantidad de información sobre el personaje en cuestión, su ámbito se diversifica lo máximo posible en formatos ajenos al cine, como videojuegos, juguetes, libros o revistas, consiguiendo así una expansión transmedia de la narrativa inherente a sus producos (Jenkins, 2009).

Gracias a los parques temáticos, los productos Disney también gozan de un espacio de cercanía con el público, ya que sus asistentes pueden interactuar de primera mano con los personajes gracias a los trabajadores del parque. De la misma forma, esta inmersión total en el universo Disney consigue crear en el público el deseo de consumir de nuevo las obras de Disney, creando un verdadero ciclo en la vida del producto que, en lugar de ser exprimido hasta que desaparece, se mantiene estable y presente en la mente de los consumidores.