\subsubsection{Desarrollo tecnológico}

Desde que se estrenó la primera película de Disney en 1928 la empresa ha sufrido
grandes cambios, siendo uno de los más notables su tecnología.

Una de las grandes claves de su continuo crecimiento es la innovación tecnológica. Cuenta con dos sedes de I+D+i en EEUU y Suiza que forman Disney Research, en la que podemos encontrar alianzas tecnológicas con Carnegie Mellon University y el Swiss Federal Institute of Technology (ETH) de Zurich, pioneras en ciencias de la computación. Sus laboratorios nacieron con el objeto de que la compañía nunca quedara desfasada.

Algunas de las innovaciones incrementales y a su vez revolucionarias llevadas a cabo por Disney son:

\begin{itemize}

\item
La introducción del sonido sincronizado en sus primeras animaciones.

\item
La cámara multiplano, permitiendo que las animaciones fueran más realistas.

\item
El efecto humo creado por ordenador.

\item
Fantasound, que permitía por primera vez escuchar sonido en estéreo.

\item
Disneyland, que aunque ya existían los parques temáticos éste concentraba distintos en un mismo lugar.

\end{itemize}

Actualmente, Disney Research está centrado principalmente en:

\begin{itemize}

\item
Robótica

\item
Inteligencia Artificial

\item
Aprendizaje automatizado

\item
Ciencias de la conducta

\item
Procesador de la imagen

\item
Interacción hombre-máquina

\item
Investigación de materiales

\end{itemize}

Podemos destacar el actual desarrollo de un programa de reconstrucción facial mediante \textit{modelización}, que consiste en un grupo de siete cámaras que captan imágenes al mismo tiempo de una persona para, posteriormente, reconstruirla en base a las imágenes grabadas. También hemos de mencionar el \textit{visible light communication system}, un sistema que pretende utilizar la luz para la transmisión de datos, lo cual supondría una innovación radical en el ámbito de las telecomunicaciones.

En cuanto a su primera patente, ésta fue la del famoso ratón Mickey Mouse, la cual oscila en este momento sobre los 3.000 millones de dólares. Curiosamente, cuando el copyright del conocido ratón está apunto de expirar se produce el Efecto Mickey Mouse, es decir, se cambia la ley de la propiedad de los derechos y registros de autoría para ampliar el plazo de posesión exclusiva de Mickey para Disney (Schlackman, 2014).


