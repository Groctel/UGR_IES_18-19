\subsubsection{Desarrollo tecnológico}

Desde que se estrenó la primera película de Disney en 1928 la empresa ha sufrido
grandes cambios, siendo uno de los más notables su tecnología.

Una de las grandes claves de su continuo crecimiento es la innovación tecnológica. Cuenta con dos sedes de I+D+i en EEUU y Suiza que forman Disney Research en la que podemos encontrar alianzas tecnológicas con: Carnegie Mellon University y el Swiss Federal Institute of Technology (ETH) de Zurich, pioneras en ciencias de la computación. Sus laboratorios nacieron con el objeto de que la compañía nunca quedara desfasada.

Algunas de las innovaciones incrementales y a su vez revolucionarias llevadas a cabo por Disney son:
\begin{itemize}
\item
\textbf{La introducción del sonido sincronizado en sus primeras animaciones.}
\item
\textbf{La cámara multiplano, permitiendo que las animaciones fueran más realistas.}
\item
\textbf{El efecto humo creado por ordenador.}
\item
\textbf{Fantasound, que permitía por primera vez escuchar sonido en estéreo.}
\item
\textbf{Disneyland, que aunque ya existían los parques temáticos éste concentraba distintos en un mismo lugar.}

\end{itemize}
Actualmente Disney Research está centrado principalmente en:
\begin{itemize}

\item
\textbf{Robótica}
\item
\textbf{Inteligencia Artificial}
\item
\textbf{Aprendizaje automatizado}
\item
\textbf{Ciencias de la conducta}
\item
\textbf{Procesador de la imagen}
\item
\textbf{Interacción hombre-máquina}
\item
\textbf{Investigación de materiales}

\end{itemize}

Podemos destacar el actual desarrollo un programa de reconstrucción facial mediante modelización que consiste en un grupo de siete cámaras que captan imágenes al mismo tiempo de una persona para, posteriormente, reconstruir su cara en base a las imágenes grabadas. También hemos de mencionar el "visible light communication system" el cual es un sistema que pretende utilizar la luz para la transmisión de datos, lo cual supondría una innovación radical.

En cuanto a su primera patente, ésta fue la del famoso ratón Mickey Mouse la cual oscila en este momento sobre los 3.000 millones de dólares. Curiosamente cuando el copyright del conocido ratón está apunto de expirar se produce el Efecto Mickey Mouse, es decir, mágicamente cambia la ley de la propiedad de los derechos y registros de autoría.


