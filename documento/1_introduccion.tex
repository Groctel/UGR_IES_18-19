\section{Introducción}
\textbf{The Walt Disney Company}, conocida por el público general como \textit{Disney}, es el mayor ejemplo de éxito en el mundo de la animación y el entretenimiento infantil. Sin embargo, su prestigio no se queda ahí, ya que se ha convertido en la mayor compañía de medios de comunicación y entretenimiento del mundo.

La empresa ha sabido llevar a cabo una diversificación por la cual ya no solo se dedica a cortos de dibujos animados, sino que también produce películas de acción con actores reales e incluso ha creado parques de atracciones, hoteles y cruceros de ocio.

También ha sido capaz de adaptarse al cambio de estética, pues poco tienen que ver las primeras animaciones del ratón Mickey Mouse con las actuales, o sus primeras princesas, como Blancanieves, con sus nuevos personajes femeninos, como Moana/Vaiana.

A lo largo de los años, The Walt Disney Company ha evolucionado y dividido su compañía en empresas subsidiarias que se encarguen de un campo en concreto. Siendo algunas de las más reconocidas Walt Disney Pictures, Marvel Entertaiment o Pixar Animation Studios entre otras.

Con una amplia gama de productos cinematográficos y servicios de entretenimiento, la marca Disney llega diariamente a personas de todo el mundo y de todas las edades, reafirmando así su posición de claro asentamiento en la industrias culturales.