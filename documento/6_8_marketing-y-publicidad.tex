\subsection{Márketing y publicidad}

The Walt Disney Company, a través del buen trabajo de su departamento de marketing, ha conseguido crear un mundo de emociones para todos los públicos. Entre las estrategias de marketing llevadas a cabo podemos señalar:

\begin{itemize}

\item
\textbf{Segmentación de la audiencia a través de una estrategia multicanal:} Disney ofrece todo tipo de contenido tanto para niños, como para adultos. Un ejemplo de esto son Star Wars y Frozen, que atraen a todas las generaciones.

\item
\textbf{Usar la nostalgia para mantener la lealtad del cliente:} Esta es una herramienta que ha producido que los adultos rememoren los viejos clásicos y a su vez inculquen a sus hijos el sentimiento de Disney. En los últimos años hemos podido ver la remasterización de películas como La Bella y la Bestia, Winnie the Pooh, Dumbo y Aladdín.

\item
\textbf{Usar el storytelling para inspirar:} Disney tiene en cuenta los problemas de la vida moderna y con las historias de sus personajes pretende enseñar valores, así como la aceptación y comprensión de los sentimientos. En la película Inside Out se combinaron las emociones con el humor para reflejarlo en las distintas situaciones que una persona puede tener a lo largo de su vida.

\item
\textbf{Concesión de licencias:} Han establecido una serie de normas con lo que se puede hacer o no con su marca para evitar que su imagen pueda ser perjudicada. Aunque el merchandaising sea producido por otras empresas los productos no dejan de desprender la esencia de Disney.

\end{itemize}

En cuanto a la publicidad, Disney se autopromociona en su propio canal de televisión, apareciendo así menos productos de otras marcas. Aunque se pierda dinero de marcas que quieran puclicitarse en sus canales, esto se compensa con el rendimiento de su publicidad autorreferencial.

En conclusión, Disney pretende sorprender a su público, generando expectación y experiencias en sus clientes.