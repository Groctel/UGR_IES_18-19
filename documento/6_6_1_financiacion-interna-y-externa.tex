\subsubsection{Financiación interna y externa}
Como podemos observar en el reporte fiscal de 2018, la financiación interna de Disney se clasifica por sus filiales y ramas. 

\begin{table}[]
\centering
\begin{tabular}{rl}
\textbf{Media Networks} & 24.500M\$ \\
 
\textbf{Parks, Experiences and Consumer Products} & 20.296M\$ \\
 
\textbf{Studio Entertainment} & 9.987M\$ \\

\textbf{Direct-to-consumer and International} & 4.651M\$ \\

\end{tabular}
\caption{\label{fig:frog} Ingresos por sector (The Walt Disney Company, 2018)}
\end{table}

El pasado ejercicio hubo una mejora generalizada en los ingresos respecto a 2017, con la única excepción de la rama Consumer Products and Interactive Media, que prepara el lanzamiento del servicio de \textit{streaming} \textquotedblleft Disney+ \textquotedblright, con la inversión que ello conlleva.

La financiación externa proviene de los accionistas. Durante este año, Disney ha mejorado considerablemente sus cifras: la mejora ha sido del 46.92\% en el EPS (ganacias por acción) anual.

\begin{table}[]
\centering
\begin{tabular}{rlrl}
\textbf{2018} & 8'36M\$ \\
 
\textbf{2017} & 5'69M\$ \\

\textbf{2016} & 5'73M\$ \\

\end{tabular}
\caption{\label{fig:frog} Earnings per share (The Walt Disney Company, 2018)}
\end{table}