\subsection{Forma de control}

\begin{itemize}

\item
\textbf{Control preventivo o preliminar:} The Walt Disney Company lleva a cabo varias formas de control preventivo en sus producciones audiovisuales. Los \textit{test screenings} son una práctica muy común en el sector consistentes en mostrar un fragmento de la película a una muestra de población para ver sus reacciones en las primeras etapas del desarrollo. Esto permite adaptar la película a las críticas recibidas. En Disneyland y otros parques se imprime diariamente un periódico exclusivo para los empleados llamado \textit{Eyes \& Ears} que contiene toda la información pertinente sobre los horarios del día, incidencias y protocolos de actuación.

\item
\textbf{Control constante o concurrente:} Destaca en los parques y resorts. La tarjeta FASTPASS, que es la llave de la habitación y actúa como tarjeta de crédito, también permite conocer los gustos del cliente durante su estancia (Marr, 2017). Esto influye en la experiencia Disney del individuo que si, por ejemplo, quisiera acudir a una atracción con largo tiempo de espera, será rápidamente redirigido por el personal a otra similar.

\item
\textbf{Control correctivo o retroalimentación:} Walt Disney siempre abogó por crear la mejor experiencia posible en sus parques, por ello tras la apertura del primer parque instauró varias políticas basadas en lo que observó y las quejas recibidas por los clientes. Por ejemplo, dedujo que los consumidores no guardaban su basura más de 10 metros hasta tirarla directamente en el suelo, así que se instalaron papeleras a esa distancia. Además, ver a personal vaciando las papeleras arruinaba la experiencia, así que se construyó un enorme sistema de tuberías automático (Ashraf, 2017).

\end{itemize}