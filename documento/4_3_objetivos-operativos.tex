\subsection{Objetivos operativos}
Los objetivos operacionales son aquellos que consideran una actividad o tarea específica, y suelen ser preparados por los jefes o responsables de cada equipo de trabajo. Además son objetivos a corto plazo.

Un ejemplo de este tipo de objetivos sería en la división de equipos que se hace en algunas películas de animación, por ejemplo en Moana se creó una división para animar el agua del mar y tomas todas las decisiones que lo involucren.

En los parques podemos coger la dirección que se hace en los espectáculos y desfilen que ocurren a determinadas horas.