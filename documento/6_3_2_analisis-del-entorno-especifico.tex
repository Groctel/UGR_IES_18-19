\subsubsection{Análisis del entorno específico}
Como se ha mencionado anteriormiente, The Walt Disney Company opera en el sector del entretenimiento y los medios de comunicación de masas. Siguiendo el modelo de Porter del diamante de cinco fuerzas, podemos encontrar las siguientes:

\begin{itemize}

\item
\textbf{Fuerza 1 - Amenaza de nuevos competidores:} The Walt Disney Company tiene previsto introducirse en el servicio del \textit{streaming}, convirtiendo a empresas como Netflix y HBO en sus nuevos competidores. En cuanto al mundo cinematográfico, existen competidores ya establecidos como DreamWorks, Nickelodeon y WarnerBros.

\item
\textbf{Fuerza 2 - Intensidad de la rivalidad entre competidores existentes:} Como se ha mencionado anteriormente, The Walt Disney Company tiene una relación simbiótica con las empresas, lo que permite eliminar gran parte de la competencia, y está tan establecida que apenas tiene competencia. En caso de tenerla, adquiere a la compañía que pueda causarles perjuicios económicos, como es el caso de Fox (Gartenberg, 2017).

\item
\textbf{Fuerza 3 - Productos sustitutos:} En este caso cualquier otra empresa dedicada al entretenimiento infantil como lo son Boing o Clan Tv es un producto sustituto de Disney. SixFlags o SeaWorld son, por otro lado, parques temáticos sustitutos de Disneyland. No obstante, las amenazas de productos sustitutos es baja debido a la distinción de la marca y su imagen.

\item
\textbf{Fuerza 4 - Poder negociador de los clientes:} Debido a la popularidad de la marca y la diferenciación de los productos que ofrece, el poder de negociación de los clientes es débil, lo cual ha incrementado la lealtad de los mismos. Esto se ha podido reflejar en la subida de precio en las entradas de sus parques temáticos y la apenas inexistente pérdida de visitante a causa de ello.

\item
\textbf{Fuerza 5 - Poder negociador de los proveedores:} La integración vertical de Disney reduce el poder negociador de los proveedores. La compañía se caracteriza por la alta calidad de sus productos, por lo que los proveedores deben de comprometerse a ofrecer productos con las características esperadas, ya que en caso contrario son rápidamente reemplazados.

\end{itemize}