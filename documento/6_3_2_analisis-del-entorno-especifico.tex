\subsubsection{Análisis del entorno específico}
Para analizar el entorno específico nos tenemos que centrar en el sector en el que trabaja y se desenvuelve la empresa, en este caso es el sector del entretenimiento y los medios de comunicación de masas.

La intensidad de la competencia en el sector se evalúa con el Diamante de Porter que depende de 5 fuerzas:

\textbf{Fuerza 1: Amenaza de nuevos competidores.} The Walt Disney company tiene previsto introducirse en el servicio de streaming, convirtiendo a empresas como Netflix y HBO en sus nuevos competidores.

\textbf{Fuerza 2: Intensidad de la rivalidad entre competidores existentes.} Como se ha mencionado en el punto anterior, Disney tiene una relación simbiótica con las empresas que hace que no compiten y está tan establecida que apenas tiene competencia. \textit{(La compra de Fox elimina este competidor de la escena)}

\textbf{Fuerza 3: Productos sustitutos.} En este caso cualquier otra empresa dedicada al entretenimiento infantil como lo son Boing o Clan Tv es un producto sustituto de Disney.

\textbf{Fuerza 4: Poder negociador de los clientes.} Aunque el número de clientes es grande.

\textbf{Fuerza 5: Poder negociador de los proveedores.}