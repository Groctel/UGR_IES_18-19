\subsubsection{Análisis del entorno específico}
Para analizar el entorno específico nos tenemos que centrar en el sector en el que trabaja y se desenvuelve la empresa, en este caso es el sector del entretenimiento y los medios de comunicación de masas.

La intensidad de la competencia en el sector se evalúa con el Diamante de Porter que depende de 5 fuerzas:

\begin{itemize}

\item
\textbf{Fuerza 1 - Amenaza de nuevos competidores:} The Walt Disney Company tiene previsto introducirse en el servicio de streaming, convirtiendo a empresas como Netflix y HBO en sus nuevos competidores. En cuanto al mundo cinematográfico podemos nombrar a DreamWorks, Nickelodeon y WarnerBros.

\item
\textbf{Fuerza 2 - Intensidad de la rivalidad entre competidores existentes:} Como se ha mencionado en el punto anterior, Disney tiene una relación simbiótica con las empresas que hace que no compiten y está tan establecida que apenas tiene competencia. \textit{(La compra de Fox elimina este competidor de la escena)}

\item
\textbf{Fuerza 3 - Productos sustitutos:} En este caso cualquier otra empresa dedicada al entretenimiento infantil como lo son Boing o Clan Tv es un producto sustituto de Disney. Y en relación a los parques temáticos SixFlags o SeaWorld. No obstante, las amenazas de productos sustitutos es baja, debido a la distinción de la marca y su imagen.

\item
\textbf{Fuerza 4 - Poder negociador de los clientes:} El poder de negociación de los clientes es débil debido a la popularidad de la marca y la diferenciación de los productos que ofrece, lo cual ha incrementado la lealtad de sus clientes. Esto se ha podido reflejar en la subida de precio en las entradas de sus parques temáticos y la a penas inexistente pérdida de clientes.

\item
\textbf{Fuerza 5 - Poder negociador de los proveedores:} La integración vertical de Disney reduce el poder negociador de los proveedores. La compañía se caracteriza por la alta calidad de sus productos, por lo que los proveedores deben de comprometerse a ofrecer productos con las características esperadas, ya que en caso contrario son rápidamente reemplazados.

\end{itemize}