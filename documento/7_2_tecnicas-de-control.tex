\subsection{Técnicas de control}
\begin{itemize}

\item\textbf{Control de presupuestos:} alinean los costes operativos de la compañía con los objetivos estratégicos. Las películas se presupuestan antes de su producción y márketing, y los  parques tienen su presupuesto para los costes de las parcelas, edificios y equipamiento, personal... Cuando se empezó a producir juguetes y otro merchandising se diseñó un cuidadoso plan presupuestario para poder controlar la entrada a este nuevo mercado.

\item\textbf{Auditorías:} supervisión de toda la empresa: sistemas de control, finanzas, aplicación de políticas y procedimientos y comprobar que todo esté dentro de la legalidad. Existe un comité específico para esto, que realiza evaluaciones periódicas con los directivos para discutir o aconsejar sobre cualquier hallazgo. Teniendo en cuenta la magnitud de The Walt Disney Company, este es un trabajo extremadamente importante y minucioso  a la hora de realizar un reporte de finanzas.

\item\textbf{Uso de estadísticas:} permite conocer las necesidades del consumidor y la eficacia de los planes llevados a cabo. Actualmente, en los Parques Disney se emplea Big Data e Internet of Things  para mantener su fama de "El lugar más mágico del mundo" con unas pulseras que, además de actuar como tarjeta de hotel y de crédito,  recaban datos de la satisfacción y posibles necesidades de los clientes.

\end{itemize}