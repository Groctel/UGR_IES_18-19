\subsubsection{Proceso de producción}
La creación de un nuevo producto en Disney comienza con un \textit{brainstorming}, una lluvia de ideas del proyecto que se va a comenzar. En esta etapa se busca encontrar aquella idea que sea la mas creativa y con ello la más diferenciada que aporte el máximo beneficio posible.

Walt Disney tenía su propio método, llamado \textit{Imagineering}, ingeniería de la imaginación, que contaba con tres fases: la del soñador (asocida al \textit{brainstorming} mencionado anteriormente), realista y crítico (Elmansy, 2015). Con ellas conseguía elegir las mejores ideas y contestar a las siguientes preguntas:

\begin{itemize}

\item
\textbf{El soñador - ¿Adónde podemos ir?:} En esta fase se deja volar la imaginación, sin tener límites y sin forzar a que una idea sea realista.

\item
\textbf{El realista - ¿Cómo podemos llegar ahí?:} Es el que se cuestiona si una idea es realizable o no, intenta comprender cómo hacer lo que en la etapa anterior hemos propuesto sin tener límite alguno.

\item
\textbf{El crítico - ¿Se puede llegar?:} Busca las debilidades y desmonta argumentos que en caso de fallar en ello indica que se trata de una buena idea, la cual puede convertirse en un producto final de éxito.

\end{itemize}

Estas fases debían de estar aisladas las unas de las otras, pues consideraba que el realismo no podía interferir en la primera fase, que era la más creativa. No obstante, existen otros factores importantes como lo son la comunidad, la colaboración y el buen hacer de los trabajadores, que influyen enormemente en este proceso.

