\subsection{Recursos humanos}
A pesar de su tamaño, The Walt Disney Company se centra siempre en el elemento humano de la organización. Disney tiene una buena reputación en lo referente al trato hacia sus trabajadores, desde la socialización recibida a un ambiente de trabajo inclusivo o una buena ética laboral .

Lee Cockerell, el anterior director de operaciones en Disney World Resort, distingue 6 principios en de la dirección de recursos humanos de Disney:

\begin{itemize}

\item
\textbf{Todo el mundo es importante:} Trata a tus trabajadores como tus iguales, respétalos y valóralos. Estáte dispuesto siempre a ayudar.

\item
\textbf{Rompe las normas:} Los cambios estructurales de Disney le han ofrecido muchas oportunidades. Integrar y alinear departamentos ofrece nuevas formas de pensar.

\item
\textbf{Haz de tu gente tu marca:} Busca siempre al candidato perfecto.

\item
\textbf{Haz magia con la formación:} La socialización y formación es un punto fuerte de Disney. Todos los trabajadores en los parques comienzan con un cursillo sobre tradiciones.

\item
\textbf{Elimina los obstáculos:} Es responsabilidad de un buen líder es saber identificar los problemas y resolverlos rápido.

\item
\textbf{Aprende:} Los grandes líderes están aprendiendo continuamente. Experimenta y aprende nuevas cosas.

\end{itemize}

Como ya hemos visto, la misión de Disney es \textquotedblleft (...) entretener, informar e inspirar a personas de todo el mundo mediante el poder de la incomparable narrativa, reflejando las marcas icónicas, mentes creativas y tecnologías innovadoras (...)\textquotedblright. \textit{Inspirar}, \textit{creatividad} o \textit{innovación} son palabras clave y ello se refleja en la forma de trabajar. Una de las filosofías que prima es "Dream as a team", se le da gran libertad a los empleados y las buenas ideas son alentadas desde todos los puestos en sesiones periódicas de lluvia de ideas.