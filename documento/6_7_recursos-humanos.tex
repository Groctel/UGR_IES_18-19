\subsection{Recursos humanos}
A pesar de su tamaño, The Walt Disney Company se centra siempre en el elemento humano de la organización. Disney tiene una buena reputación respecto a cómo trata a sus trabajadores, desde la socialización recibida, un ambiente de trabajo inclusivo  o una buena ética laboral .

Lee Cockerell, el anterior director de operaciones en Disney World Resort, distingue 6 principios en de la dirección de recursos humanos de Disney:
\begin{itemize}
\item Todo el mundo es importante: trata a tus trabajadores como tus iguales, respétalos y valóralos. Estáte dispuesto siempre a ayudar.

\item Rompe las normas: los cambios estructurales de Disney le han ofrecido muchas oportunidades. Integrar y alinear departamentos ofrece nuevas formas de pensar.

\item Haz de tu gente tu marca: busca siempre al candidato perfecto.

\item Haz magia con la formación: la socialización y formación es un punto fuerte de Disney. Todos los trabajadores en los parques comienzan con un cursillo sobre tradiciones.

\item Elimina los obstáculos: es responsabilidad de un buen líder es saber identificar los problemas y resolverlos rápido.

\item Aprende: los grandes líderes están aprendiendo continuamente. Experimenta y aprende nuevas cosas.

\end{itemize}

Como ya hemos visto, la misión de Disney es \textit{"(...) entretener, informar e inspirar a personas de todo el mundo mediante el poder de la incomparable narrativa, reflejando las marcas icónicas, mentes creativas y tecnologías innovadoras (...)"}. Inspirar, creatividad o innovación son palabras clave, y se refleja en la forma de trabajar. Una de las filosofías que prima es "Dream as a team", se les da gran libertad a los empleados y las buenas ideas son alentadas desde todos los puestos en sesiones periódicas de lluvia de ideas.
